\documentclass[11pt,a4paper]{article}

\usepackage{amsmath,amssymb,graphicx}
\usepackage{hyperref}
\usepackage{geometry}
\geometry{margin=2.5cm}

\title{Projet : Reconstruction 3D à partir de Projections 2D \\ 
{\large Rapport d'avancement}}
\author{---}
\date{\today}

\begin{document}

\maketitle

\section{Introduction}

Ce projet s'inscrit dans le cadre du module “Reconstruction 3D à partir de projections 2D”
proposé à l’INSA. L’objectif global est d’explorer les méthodes mathématiques permettant
de reconstruire une forme à partir de projections partielles, depuis la tomographie médicale
jusqu’aux illusions d’optique telles que les \emph{ambiguous cylinders} de Sugihara 
\cite{Sugihara}.  

Dans ce rapport, nous présentons l’ensemble du travail réalisé jusqu’à présent :
\begin{itemize}
    \item étude de la transformée de Radon et construction de sinogrammes,
    \item implémentation de la rétroprojection filtrée,
    \item implémentation d’un algorithme ART,
    \item premières recherches sur les objets ambigus de Sugihara.
\end{itemize}

\section{Transformée de Radon}

\subsection{Définition}

La transformée de Radon d’une image $f(x,y)$ consiste à intégrer cette image le long de
droites paramétrées par un angle $\theta$ et une distance $\rho$ :
\[
R f(\rho,\theta) = \int_{-\infty}^{\infty} f(x \cos \theta - y \sin\theta + \rho,\,
x \sin\theta + y \cos\theta)\, dx.
\]

\subsection{Construction du sinogramme}

Nous avons implémenté une version discrète de la transformée :  
pour chaque angle $\theta$ et pour chaque ligne projetée, nous \textbf{sommmeons les pixels rencontrés}.  
Le résultat est stocké dans une matrice $(\rho, \theta)$ formant le sinogramme.

\begin{itemize}
    \item boucle sur les angles $\theta$,
    \item pour chaque angle, rotation de l’image ou interpolation,
    \item somme des intensités pixel par pixel,
    \item stockage dans la matrice sinogramme.
\end{itemize}

Nous avons vérifié visuellement le bon comportement du sinogramme à l’aide de motifs simples
(disque, carré, lettres).

\section{Rétroprojection Filtrée}

À partir du sinogramme, nous avons implémenté la reconstruction par \textbf{rétroprojection filtrée}
(FBP — Filtered Back Projection).  
La méthode se décompose en deux étapes :

\begin{enumerate}
    \item \textbf{Filtrage de Ram-Lak} : convolution de chaque projection avec un filtre haute fréquence.
    \item \textbf{Rétroprojection} : pour chaque angle $\theta$, redistribution des valeurs filtrées 
    le long des lignes correspondantes dans l’espace image.
\end{enumerate}

Cette méthode fournit une reconstruction satisfaisante mais sensible au bruit, ce que l’on observe déjà sur nos tests.

\section{Méthode ART (Algebraic Reconstruction Technique)}

Nous avons également codé une version simple de l’algorithme ART, qui repose sur la mise à jour
itérative de l’image $x$ par :

\[
x^{(k+1)} = x^{(k)} + \lambda \frac{p_i - \langle a_i, x^{(k)} \rangle}{\|a_i\|^2} a_i
\]

où $p_i$ est la $i$-ème projection et $a_i$ la ligne correspondante de la matrice système.

ART converge lentement mais donne souvent des résultats plus lisses que la FBP.

\section{Ambiguous Cylinders : premières recherches}

Nous avons commencé à étudier les objets ambigus de Sugihara, en particulier :

\begin{itemize}
    \item visionnage et analyse de vidéos explicatives,
    \item étude des contraintes géométriques : projections orthographiques différentes créant des silhouettes distinctes,
    \item compréhension que ces objets reposent sur une approximation géométrique stricte,
    basée sur une forme 3D soigneusement déformée.
\end{itemize}

À ce stade, nous avons commencé à relier la problématique à la tomographie :  
\emph{les silhouettes imposées peuvent être interprétées comme des projections contraintes}.

\section{Conclusion}

Nous disposons désormais :
\begin{itemize}
    \item d’un pipeline complet : Image $\rightarrow$ Sinogramme $\rightarrow$ Reconstruction (FBP et ART),
    \item d’une bonne compréhension des illusions de Sugihara,
    \item des bases nécessaires pour la suite : optimisation de formes 3D sous contraintes de projection.
\end{itemize}

Les prochaines étapes incluront l’optimisation numérique d’une forme produisant deux silhouettes
distinctes selon deux angles de vue.

\begin{thebibliography}{9}

\bibitem{Sugihara}
K. Sugihara, \emph{Ambiguous cylinders: A new class of impossible objects}, 
Computer Aided Geometric Design, 2016.

\bibitem{Kak}
A. C. Kak, M. Slaney, \emph{Principles of Computerized Tomographic Imaging}, SIAM, 2001.

\bibitem{Herman}
G. Herman, \emph{Fundamentals of Computerized Tomography}, Springer, 2009.

\end{thebibliography}

\end{document}